% UG project example file, February 2022
%   A minior change in citation, September 2023 [HS]
% Do not change the first two lines of code, except you may delete "logo," if causing problems.
% Understand any problems and seek approval before assuming it's ok to remove ugcheck.
\documentclass[logo,bsc,singlespacing,parskip]{infthesis}
\usepackage{ugcheck}
\setlength{\parindent}{0pt}
\usepackage{graphicx} % Required for inserting images
\usepackage{minted}
\usepackage{float}  % 加载float包


% Include any packages you need below, but don't include any that change the page
% layout or style of the dissertation. By including the ugcheck package above,
% you should catch most accidental changes of page layout though.

\usepackage{microtype} % recommended, but you can remove if it causes problems
\usepackage{natbib} % recommended for citations

\usepackage{listings}
\usepackage{xcolor}

\lstset{
    language=C++,                % 设置代码语言
    basicstyle=\ttfamily,        % 基本字体样式
    keywordstyle=\color{blue},   % 关键字颜色
    stringstyle=\color{red},     % 字符串颜色
    commentstyle=\color{green},  % 注释颜色
    numbers=left,                % 显示行号
    numberstyle=\tiny\color{gray}, % 行号的样式
    stepnumber=1,                % 每行显示行号
    breaklines=true,             % 自动换行
    tabsize=4              % Tab 宽度
}





\begin{document}
\begin{preliminary}

\title{This is the Project Title}

\author{Jason Yu}

% CHOOSE YOUR DEGREE a):
% please leave just one of the following un-commented
\course{Artificial Intelligence}
%\course{Artificial Intelligence and Computer Science}
%\course{Artificial Intelligence and Mathematics}
%\course{Artificial Intelligence and Software Engineering}
%\course{Cognitive Science}
%\course{Computer Science}
%\course{Computer Science and Management Science}
%\course{Computer Science and Mathematics}
%\course{Computer Science and Physics}
%\course{Software Engineering}
%\course{Master of Informatics} % MInf students

% CHOOSE YOUR DEGREE b):
% please leave just one of the following un-commented
%\project{MInf Project (Part 1) Report}  % 4th year MInf students
%\project{MInf Project (Part 2) Report}  % 5th year MInf students
\project{4th Year Project Report}        % all other UG4 students


\date{\today}

\abstract{
This skeleton demonstrates how to use the \texttt{infthesis} style for
undergraduate dissertations in the School of Informatics. It also emphasises the
page limit, and that you must not deviate from the required style.
The file \texttt{skeleton.tex} generates this document and should be used as a
starting point for your thesis. Replace this abstract text with a concise
summary of your report.
}

\maketitle

\newenvironment{ethics}
   {\begin{frontenv}{Research Ethics Approval}{\LARGE}}
   {\end{frontenv}\newpage}

\begin{ethics}
\textbf{Instructions:} \emph{Agree with your supervisor which
statement you need to include. Then delete the statement that you are not using,
and the instructions in italics.\\
\textbf{Either complete and include this statement:}}\\ % DELETE THESE INSTRUCTIONS
%
% IF ETHICS APPROVAL WAS REQUIRED:
This project obtained approval from the Informatics Research Ethics committee.\\
Ethics application number: ???\\
Date when approval was obtained: YYYY-MM-DD\\
%
\emph{[If the project required human participants, edit as appropriate, otherwise delete:]}\\ % DELETE THIS LINE
The participants' information sheet and a consent form are included in the appendix.\\
%
% IF ETHICS APPROVAL WAS NOT REQUIRED:
\textbf{\emph{Or include this statement:}}\\ % DELETE THIS LINE
This project was planned in accordance with the Informatics Research
Ethics policy. It did not involve any aspects that required approval
from the Informatics Research Ethics committee.

\standarddeclaration
\end{ethics}


\begin{acknowledgements}
Any acknowledgements go here.
\end{acknowledgements}


\tableofcontents
\end{preliminary}


\chapter{Introduction}

The preliminary material of your report should contain:
\begin{itemize}
\item
The title page.
\item
An abstract page.
\item
Declaration of ethics and own work.
\item
Optionally an acknowledgements page.
\item
The table of contents.
\end{itemize}

As in this example \texttt{skeleton.tex}, the above material should be
included between:
\begin{verbatim}
\begin{preliminary}
    ...
\end{preliminary}
\end{verbatim}
This style file uses roman numeral page numbers for the preliminary material.

The main content of the dissertation, starting with the first chapter,
starts with page~1. \emph{\textbf{The main content must not go beyond page~40.}}

The report then contains a bibliography and any appendices, which may go beyond
page~40. The appendices are only for any supporting material that's important to
go on record. However, you cannot assume markers of dissertations will read them.

You may not change the dissertation format (e.g., reduce the font size, change
the margins, or reduce the line spacing from the default single spacing). Be
careful if you copy-paste packages into your document preamble from elsewhere.
Some \LaTeX{} packages, such as \texttt{fullpage} or \texttt{savetrees}, change
the margins of your document. Do not include them!

Over-length or incorrectly-formatted dissertations will not be accepted and you
would have to modify your dissertation and resubmit. You cannot assume we will
check your submission before the final deadline and if it requires resubmission
after the deadline to conform to the page and style requirements you will be
subject to the usual late penalties based on your final submission time.

\section{Using Sections}

Divide your chapters into sub-parts as appropriate.

\section{Citations}

Citations, such as \citet{P1} or \citep{P2}, can be generated using
\texttt{BibTeX}. We recommend using the \texttt{natbib} package (default) or the newer \texttt{biblatex} system. 

You may use any consistent reference style that you prefer, including ``(Author, Year)'' citations. 

\chapter{Background Chapter}

\section{Overview}



Libseff is a library that designed to provide an effect handler functionality directly to a C programmer. Effect handler is a mechanism that used to handle side effects in a program. These side effects happens when a program that interact with the external environment or change state during computation, e.g., I/O, exception handling, concurrency, state changes, etc. However, there already has two C effect handler libraries, libhandler[1] and libmpeff[2]. The key difference is, they are mainly designed for compiler developers and use as compilation targets for high-level programming languages rather than as tools for C programmers to use directly. In other words, these two libraries are more suitable for use in compiler generated code, not for C programmers to use directly or to call. And the difference between traditional effect handler implmentation is libseff use mutable coroutine object rather than immutable continuation object this means when executing effect operation we don't need to allocate new continuation object every time. Since there is no need to frequent create a new continuation object means that the cost of allocate memory is reduced, this improves the performance and reduced the memory usage[3].

A coroutine is a function that can resume execution while retaining its state, allowing for non-blocking, asynchronous programming. This makes coroutines useful for managing tasks like I/O operations, concurrency, and iterative workflows without blocking other parts of a program. They enable cooperative multitasking by yielding control at designated points, unlike threads which require context switching.

Continuation is a concept in programming, that represent the next step of the current computation work, the program can capture the current execution status and resume execution in any point by using continuation. First-class means this concept can be just like a normal value that can be passed, stored and manipulated. 
\medskip

\section{Effect handlers}

The figure 2.1 shows a simple implementation of using effect handlers.

\begin{figure}[H]
    \centering
    \includegraphics[width=0.7\textwidth]{image.png}
    \caption{Read File Code}
    \label{fig:enter-label}
\end{figure}

\begin{lstlisting}
DEFINE_EFFECT (read_file, 0, char*, {const char* filename; });
\end{lstlisting}

Here we define an effect called \texttt{read\_file}, it has tag 0 and takes \texttt{const char*} as a parameter type, which in this case is a filename. The return type is \texttt{char*}, representing the content of the file. This defines the form and interface of the \texttt{read\_file} effect.

\begin{lstlisting}
void deal_loop (seff_coroutine_t *temp)
\end{lstlisting}

Here we use \texttt{deal\_loop} to handle the effects based on the requests made by the coroutine. The \texttt{seff\_resume} function is used to resume the coroutine, where \texttt{temp} is the coroutine object. The effect that the coroutine has requested is contained in the \texttt{seff\_request\_t} object returned by the \texttt{seff\_resume} function. We use a \texttt{CASE\_EFFECT} branch to handle the \texttt{read\_file} effect once it is identified. In this case, we simulate the read file operation, not actually read the file. Then, the coroutine is resumed with the text "haloooooooooooo" passed as the result to the coroutine, and the loop will exit when the \texttt{CASE\_RETURN} branch has handled the coroutine's completion.

\begin{lstlisting}
void* read_print (void* param)
\end{lstlisting}

We create a coroutine function called \texttt{read\_print}. What it does is try to perform the \texttt{read\_file} effect, which is to request a file operation to read the content from "example.text" and print it, then return.

\begin{lstlisting}
int main(void)
\end{lstlisting}

This main function first creates a new coroutine called \texttt{k}, passing in the \texttt{read\_print} function, and then lets \texttt{deal\_loop} handle all the effect requests from this coroutine until it completes.

\medskip




This simple demonstration shows how to use an effect handler to handle effects in the coroutine. A coroutine can stop itself and request an effect through the effect handler (e.g., \texttt{read\_file}), and then handle it outside, like in \texttt{deal\_loop}, which will give the result and resume the execution of the coroutine. This design is similar to an asynchronous mechanism, completing asynchronous tasks without blocking the main program.

\medskip
\medskip

\section{Stack Management}

Stack management is one of the most important concept in libseff  library, to be more specific we need to decide how to allocated stack frame and resized it. Stack frames are the memory space used by function calls at program runtime, and how these stack frames are managed in a coroutine implementation is a key issue. In a other word different stack management strategies can result in different performance and memory usage of the program. In libseff library there already exists two stack management strategies which is Fixed-Size Stacks and Segmented Stacks which can be selected individually.



\medskip

\subsection{Fixed-Size Stack}


Fixed-Size Stack is the most simple method to manage a stack as it don't have to resize the stack dynamically, the only thing it has to do is allocate a fixed size block of the memory for coroutine when it has been created. The advantage of this approach is it's easy to implement also there is no additional runtime cost as the size of the stack is fixed and there is no need to adjust the memory during the operation. but there is a trade-off, we have mentioned that the size of the stack is fixed so that could cause memory wasted if the actual needed memory is less than the allocated memory that means the part of the unused memory will be wasted. and is hard to predict the exact memory needed if allocated memory is too small it could cause a stack overflow, if it was overly allocated that will waste large amount of memory, but to avoid the stack overflow, we normally give out memory way more than it's actual need to avoid the stack overflow increase the risk of memory wastage.



\medskip

\subsection{Segmented Stack}

Segmented stack is designed to achieve first-class continuation more efficiently and it has a call stack structure[4]. The concept of the segmented stack is change fixed size stack to linked multi stack segment or "stackelets" to manage the stack, and each function will check if the current stack has enough space for the new stack frame at runtime, if not, the system will allocate a new stack segment means that it has dynamic stack size, which reduce the risk of stack overflow and improve flexibility by using morestack function. although there are many advantage of using segmented stack but there are some previously unknown problems which may be the one of the reason why it have poor memory efficiency and performance in the past, there also has a well-known problem for the segmented stack known as "hot-split"[6,7,8]

\medskip

"hot-split" is a known problem related to segmented stack. when there is a tight function call loop, frequent allocation and release of new stack segments will significantly reduce performance[5]. Where libseff gives a optimisation, in order to reduce the problem of "hot-split" libseff using a doubly-linked list to managed a stack segment. When a stack segment is no longer needed, it will not release immediately, it will kept in a linked table for reuse to avoiding frequent memory allocations. In this way, even in the worst case, the overhead of a function call is only 11 times the a normal function call, 11 times may seem like a lot of overhead, but in practice it's not that significant, because small functions are in-lined by the compiler[9].

\medskip

The interoperability of segmented stacks and library functions may cause a problem. Because the segmented stack depends on the stack overflow check of the function, and the standard library function is usually pre-compiled and does not contain these checks, it may lead to stack overflow or silent memory corruption, to avoid this problem, the compiler will reserve more stack space when calling functions that are not supported by segmented stacks, but this will also lead to larger memory consumption[3]. libseff fix this issue by using MAKE-SYSCALL-WRAPPER macro, this macro generates a wrapper for functions, ensuring that when such functions are called, the program switches to the system stack instead of allocating a new stack segment or "stacklet". This avoids unnecessary allocation of memory.



\medskip

\subsection{Over-committing}

In the previously mentioned approach, fixed-size stacks are easy to implement and have minimal runtime costs but are significantly inefficient in memory usage, particularly when memory requirements are unpredictable. This often leads to over-allocating memory to prevent stack overflow, resulting in considerable resource waste. On the other hand, segmented stacks offer a more flexible solution by dynamically allocating memory as needed. However, they incur additional costs, especially when "hot splits" occur, which can cause substantial slowdowns. Given these limitations, it is preferable to extend a new method with a more efficient approach to stack management, known as over-committing, which provides greater flexibility in memory allocation without the associated performance costs.

\medskip

Over-committing avoids the trade-off of both fixed-size and segmented stacks by leaving it to operating system's virtual memory management. Instead of allocating a large amount of memory or frequently allocating new stack segments, over-committing gives a large virtual address space for the stack but only commits to physical memory when it is actually needed. This ensures efficient memory usage without having a risk of stack overflow and avoids cost of segmented stack's frequent allocations.  

\medskip

By implementing over-committing in libseff library, we can achieve the flexibility to handle dynamically stack sizes without human intervention, and the efficiency to avoid the unnecessary memory allocation and performance cost seen in segmented stacks. This approach ensures that stack space is used as efficiently as possible.







\clearpage

\chapter{Conclusions}

\section{Final Reminder}

The body of your dissertation, before the references and any appendices,
\emph{must} finish by page~40. The introduction, after preliminary material,
should have started on page~1.

You may not change the dissertation format (e.g., reduce the font size, change
the margins, or reduce the line spacing from the default single spacing). Be
careful if you copy-paste packages into your document preamble from elsewhere.
Some \LaTeX{} packages, such as \texttt{fullpage} or \texttt{savetrees}, change
the margins of your document. Do not include them!

Over-length or incorrectly-formatted dissertations will not be accepted and you
would have to modify your dissertation and resubmit. You cannot assume we will
check your submission before the final deadline and if it requires resubmission
after the deadline to conform to the page and style requirements you will be
subject to the usual late penalties based on your final submission time.

% \bibliographystyle{plain}
\bibliographystyle{plainnat}
\bibliography{}
[1] Daan Leijen. 2019. libhandler. https://github.com/koka-lang/libhandler.

[2] Daan Leijen and KC Sivamarakrishnan. 2023. libmprompt and libmpeff. https://github.com/koka-lang/libmprompt.

[3] Mario Alvarez-Picallo, Teodoro Freund, Dan R. Ghica, and Sam Lindley. High-level effect handlers for C (https://homepages.inf.ed.ac.uk/slindley/papers/libseff-draft-april2024.pdf)

[4] Robert Hieb, R Kent Dybvig, and Carl Bruggeman. 1990. Representing control in
the presence of first-class continuations. ACM SIGPLAN Notices (1990).

[5] Daniel Morsing. 2014. How Stacks are Handled in Go. https://blog.cloudflare.com/how-stacks-are-handled-in-go/.

[6] Golang. 2014. Switch to contiguous stacks. https://docs.google.com/document/d
/1wAaf1rYoM4S4gtnPh0zOlGzWtrZFQ5suE8qr2sD8uWQ/pub

[7] Golang. 2014. Switch to contiguous stacks. https://agis.io/post/contiguous-stacks-
golang/

[8] Rust. 2013. Abandoning segmented stacks. https://mail.mozilla.org/pipermail/rust-
dev/2013-November/006314.html

[9] Zhiyao Ma and Lin Zhong. 2023. Bringing Segmented Stacks to Embedded Systems. In Proceedings of the 24th
International Workshop on Mobile Computing Systems and Applications, HotMobile 2023, Newport Beach, California,
February 22-23, 2023. ACM, 117–123. https://doi.org/10.1145/3572864.3580344







\appendix

\chapter{First appendix}

\section{First section}

Any appendices, including any required ethics information, should be included
after the references.

Markers do not have to consider appendices. Make sure that your contributions
are made clear in the main body of the dissertation (within the page limit).

\chapter{Participants' information sheet}

If you had human participants, include key information that they were given in
an appendix, and point to it from the ethics declaration.

\chapter{Participants' consent form}

If you had human participants, include information about how consent was
gathered in an appendix, and point to it from the ethics declaration.
This information is often a copy of a consent form.


\end{document}
